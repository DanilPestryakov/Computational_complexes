\documentclass[../body.tex]{subfiles}
\begin{document}

Дана ИСЛАУ \textbf{A}x=\textbf{b}:
\begin{equation}
	   \\\textbf{A} = \begin{pmatrix}
        [3;4]& [5;6]\\
        [-1;1]& [-3;1]
        \end{pmatrix}, \textbf{b} = \begin{pmatrix}
        [-3;4]\\
        [-1;2]
        \end{pmatrix}
        \label{SLAU}
	\end{equation}
\\Известно, что субдиффиренциальный метод Ньютона расходится для нее при любом релаксационном параметре $\tau\in$[0;1]. При изменении элементов матрицы или свободной части решения все же существуют.
\\Требуется:
\\1. Максимально приблизить изменённые элементы к исходным.
\\2. Найти границы, где решение перестает существовать.
\end{document}

