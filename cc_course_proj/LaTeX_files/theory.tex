\documentclass[../body.tex]{subfiles}
\begin{document}

Сначала попробуем подобрать новый вектор $\textbf{b}$ при неизменной матрице, а после попробуем подправить матрицу $\textbf{A}$ при исходной свободной части.
\\Эвристика метода: в каждом случае возьмем средние. После составим сетку радиусов с шагом 0.1 и будем последовательно ее обходить. Воспользуемся метриками, которые будут представлены ниже, для оценки "расстояния" между новыми и исходными данными. Далее, выбрав оптимальные с данной точки зрения данные, будем стараться максимально приближать полученные интервалы к исходным, равномерно проходя по сетке в обе стороны интервалов уже с шагом сетки 0.1.
\\Таким образом решение состоит из двух этапов:
\\1. Сначала строим интервалы с исходными центрами, но меньшими радиусами
\\2. Максимально "растягиваем" интервалы в обе стороны
\end{document}